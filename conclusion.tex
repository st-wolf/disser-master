\conclusion

В заключении перечислим основные результаты диссертационной работы.
%
\begin{itemize}
\item[1.] Метод кодировки решений, разработанный в работе \cite{AlfAvr} для уравнения Гросса-Питаевского, был обобщен на более широкий класс потенциалов, а именно на случай знакопеременного нелинейного потенциала в уравнении (\ref{eq:GPE}).
\item[2.] Были получены строгие утверждения о границах применимости метода кодировки.
\item[3.] Было обнаружено ранее неизвестное устойчивое локализованное стационарное решение уравнения (\ref{eq:GPE}), которое получило название <<дипольный солитон>>.
\item[4.] Подробно исследовано семейство решений типа дипольный солитон (DS), описана бифуркация этого семейства по параметру $\omega$.
\item[5.] Устойчивость дипольного солитона была исследована с помощью спектрального метода и подтверждена моделирование его эволюции.
\end{itemize}
%

