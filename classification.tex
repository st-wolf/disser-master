\chapter{Классификация}

Предположим, что $Q(x)$, $P(x)$ в уравнении (\ref{eq:stationary}) четные $\pi$-периодические функции.
Определим {\it отображение Пуанкаре} $T: \mathbb{R}^2 \to \mathbb{R}^2$, связанное с уравнением (\ref{eq:stationary}) следующим образом:
%
\begin{equation}
T
\begin{pmatrix}
u_0 \\
u_0'
\end{pmatrix}
=
\begin{pmatrix}
u(\pi) \\
u_x(\pi)
\end{pmatrix},
\end{equation}
%
где $u(x)$ --- решение задачи Коши для уравнения (\ref{eq:stationary}) с начальными условиями
%
\begin{equation}
u(0) = u_0, \quad u_x(0) = u_0'.
\label{eq:initial}
\end{equation}
%
Назовем {\it орбитой} последовательность точек $\{ p_n \}$, $p_n \in \mathbb{R}^2$ (последовательность может быть как конченой, так и бесконечной) такую, что $Tp_n = p_{n+1}$.

Определим множества $\mathcal{U}_L^+ \in \mathbb{R}^2$ и $\mathcal{U}_L^-$, $L > 0$ следующим образом: $p = (u_0, u_0') \in \mathcal{U}_L^+$ тогда и только тогда, когда решение задачи Коши для уравнения (\ref{eq:stationary}) с начальными условиями (\ref{eq:initial}) не коллапсирует на промежутке $[0;L]$.
Аналогичным образом определим $\mathcal{U}_L^-$ как множество начальных условий (\ref{eq:initial}) таких, что соответствующее решение задачи Коши для уравнения (\ref{eq:stationary}) не коллапсирует на промежутке $[-L;0]$.
Легко показать, что отображение Пуанкаре $T$ определено только на множестве $\mathcal{U}_{\pi}^+$ и переводит его в множество $\mathcal{U}_{\pi}^-$.
Аналогично, обратное отображение $T^{-1}$ определено только на $\mathcal{U}_{\pi}^-$, и $T \mathcal{U}_{\pi}^- = \mathcal{U}_{\pi}^+$.

Далее, рассмотрим следующую последовательность множеств:
%
\begin{eqnarray*}
&& \Delta_0 = \mathcal{U}_{\pi}^+ \cap \mathcal{U}_{\pi}^-; \\
&& \Delta_{n+1}^+ = T \Delta_n^+ \cap \Delta_0, \quad n = 0,1,\dots; \\
&& \Delta_{n-1}^- = T^{-1} \Delta_n^- \cap \Delta_0, \quad n = 0,1,\dots.
\end{eqnarray*}
%
Очевидно, что $\Delta_0$ состоит из точек, имеющих $T$-образ и $T$-прообраз.
Также верными являются следующие утверждения:
%
\begin{eqnarray*}
&& \{ p \in \Delta_n^+ \} \iff \{ Tp, T^2p, \dots, T^np \in \Delta_0 \}; \\
&& \{ p \in \Delta_n^- \} \iff \{ T^{-1}p, T^{-2}p, \dots, T^{-n}p \in \Delta_0 \}.
\end{eqnarray*}
%
Из этого следует, что множества $\Delta_n^{\pm}$ образуют последовательности по включению вида
%
\begin{eqnarray*}
&& \ldots \subset \Delta_{n+1}^+ \subset \Delta_n^+ \ldots \subset \Delta_1^+ \subset \Delta_0; \\
&& \ldots \subset \Delta_{n+1}^- \subset \Delta_n^- \ldots \subset \Delta_1^- \subset \Delta_0.
\end{eqnarray*}
%
Теперь определим множества $\Delta^{\pm}$ следующим образом:
%
\begin{equation*}
\Delta^+ = \bigcap \limits_{n=1}^{\infty} \Delta_n^+, \quad \Delta^- = \bigcap_{n=1}^{\infty} \Delta_n^-.
\end{equation*}
%
Рассмотрим множество $\Delta = \Delta^+ \cap \Delta^-$.
Оно инвариантно относительно действия отображения $T$.
При этом орбиты, порождаемые точками из $\Delta$, находятся во взаимно-однозначном соответствии с регулярными решениями уравнения (\ref{eq:stationary}).
Аналитическое исследование множеств $\Delta_n^{\pm}$, $\Delta^{\pm}$ крайне сложно, однако возможно численное построение.
Такое построение позволяет предсказать и численно построить решения уравнения (\ref{eq:stationary}).

Утверждения \ref{prop:continuation}, \ref{prop:singular} из предыдущей главы накладывают ограничения на функции $Q(x)$, $P(x)$.
В частности, если $Q(x)$, $P(x)$ ограниченные, периодические, и $P(x) > 0$ $\forall x \in \mathbb{R}$, тогда все решения уравнения (\ref{eq:stationary}) регулярны, и наш подход не может быть применен.
В случае же когда $P(x) < 0$, $Q(x) < 0$, уравнение (\ref{eq:stationary}) не имеет несингулярных решений, за исключением нулевого, следовательно подход также не может быть применен.
Однако, как следует из Утверждения \ref{prop:asymptotic}, если $P(x)$ знакопеременная, то уход на бесконечность к конечной точке является типичным для решений уравнения (\ref{eq:stationary}), и применение нашего подхода для поиска несингулярных решений оправдано.
В работе (\ref{AlfAvr}) для случая $P(x) \equiv -1$ было показано, что если
%
\begin{itemize}
\item[(а)] множество $\Delta_0$ состоит из конченого числа $N$ компонент связности, $$
\item[(б)]
\item[(в)]
\end{itemize}
%

У нас тут должны быть ограничения --- следствия из теорем. То все сингулярно, а то ничего.