\chapter{Регулярные и сингулярные решения}

В этой главе представлены некоторые результаты аналитического исследования уравнения (\ref{eq:stationary}) для стационарных состояний.
Для дальнейшего условимся о следующей терминологии.
Решение $u(x)$ уравнения (\ref{eq:stationary}) будем называть {\it сингулярным}, если для некоторой конечной точки $x_0 \in \mathbb{R}$ выполняется соотношение
%
$$\lim \limits_{x \to x_0} u(x) = \infty.$$
%
При этом мы будем говорить, что решение $u(x)$ {\it коллапсирует} в точке $x_0$.
Соответственно, решение $u(x)$ уравнения (\ref{eq:stationary}), не коллапсирующее ни в какой точке $\mathbb{R}$, будем называть {\it несингулярным} или {\it регулярным}.

Дальнейшие утверждения показывают, что ...

\section{$P(x) \ge P_0 > 0$: отсутствие сингулярных решений}

В случае, когда $P(x)$ --- положительная функция, при некоторых других ограничениях оказывается справедливым следующее утверждение.

\begin{proposition}
Пусть $\forall x \in \mathbb{R}$, функции $Q(x), P(x) \in C^1(\mathbb{R})$, причем
%
\begin{itemize}
\item[a)] $P(x) \ge P_0 > 0$, $|P'(x)| \le \widetilde{P};$
\item[б)] $Q(x) \ge Q_0$, $|Q'(x)| \le \widetilde{Q};$
\end{itemize}
%
тогда решение задачи Коши для уравнения (\ref{eq:stationary}) с произвольными начальными условиями $u(x_0) = u_0$, $u_x(x_0) = u_0'$ может быть продолжено на всю действительную ось $\mathbb{R}$.
\label{prop:continuation}
\end{proposition}
%
\begin{proof}
По теореме существования решений для ОДУ, существует такой интервал $\Delta = [x_0; x_1)$, что решение задачи Коши $u(x)$ уравнения (\ref{eq:stationary}) с начальными условиями $u(x_0) = u_0$, $u_x(x_0) = u_0'$ на нем существует, единственно и $u(x) \in C^2(\Delta)$.
Предположим, что $[x_0; x_1)$ --- это максимальный интервал существования решения, т.е. решение задачи Коши не может быть продолжено за точку $x = x_1$.
Умножив исходное уравнение на $4u_x$ и проинтегрировав в пределах $[x_0; x)]$, $x < x_1$, получим
%
\begin{eqnarray}
2u_{x}^2(x) + 2Q(x)u^2(x) - 2{\int \limits_{x_0}^x Q'(\xi)u^2(\xi)d\xi} + P(x)u^4(x) - \\
\nonumber - {\int \limits_{x_0}^x P'(\xi)u^4(\xi)d\xi} = 2(u_0')^2 + 2Q(x_0)u_0^2 + P(x_0)u_0^4 \equiv C,\label{eq:integrated}
\end{eqnarray}
%
Где $C$ --- некоторая известная постоянная.
Отбрасывая $u_x^2(x) \ge 0$ в левой части равенства, а также пользуясь ограниченностью снизу функций $Q(x)$ и $P(x)$ значениями $Q_0$ и $P_0$ соответственно, приходим к неравенству
%
\begin{equation}
2Q_0 u^2(x) + P_0 u^4(x) \le C + 2{\int \limits_{x_0}^x
Q'(\xi)u^2(\xi)d\xi} + {\int \limits_{x_0}^x P'(\xi)u^4(\xi)d\xi}.
\label{eq:bound}
\end{equation}
%
Производные $Q'(\xi)$ и $P'(\xi)$ заменим их оценками сверху: $Q'(\xi) \le \widetilde{Q}$, $P'(\xi) \le \widetilde{P}$, где $\widetilde{Q} \ge 0$, $\widetilde{P} \ge 0$.
Умножив обе части неравенства на $P_0 > 0$? получим, что
%
\begin{equation}
2Q_0 P_0 u^2(x) + P_0^2 u^4(x) \le P_0 C + 2P_0 \widetilde{Q}{\int\limits_{x_0}^x u^2(\xi)d\xi} + P_0 \widetilde{P}{\int \limits_{x_0}^x u^4(\xi)d\xi}.
\end{equation}
%
Обозначим $v(x) = (P_0 u^2(x) + Q_0)^2$, $v(x) \ge 0$.
Тогда
%
\begin{equation}
v(x) \le \widetilde{C} +  \dfrac{\widetilde{P}}{P_0}\int \limits_{x_0}^x w(v(\xi))~d\xi. \label{eq:tov}
\end{equation}
%
Здесь $\widetilde{C} = P_0 C + Q_0^2 \ge 0$, $\alpha = {2\widetilde{Q} P_0}/{\widetilde{P}} \ge 0$, а $w(v)$ определяется формулой
%
\begin{equation}
w(v)\equiv\alpha (\sqrt{v} - Q_0) + (\sqrt{v} - Q_0)^2.
\label{eq:defw}
\end{equation}
%
Введем в рассмотрение функцию
%
\begin{equation}
G(s) = \int \limits_{s_0}^s \dfrac{dv}{w(v)}.
\label{eq:G}
\end{equation}
%
Здесь $s_0 \ge Q_0^2$ --- произвольная постоянная, $s \ge s_0$.
Так как $w(v)$ положительна и монотонно возрастает, а интеграл
%
\begin{equation}
\int \limits_{s_0}^{+\infty} \dfrac{dv}{w(v)}
\end{equation}
%
расходится, функция $G(s)$ является положительной, монотонно возрастающей и неограниченной.
Это означает, что обратная функция $G^{-1}(r)$ определена при $r \ge 0$, монотонно возрастает и неограничена.
Сказанное позволяет применить к (\ref{eq:tov}) неравенство Бихари, \cite{Pachpatte}, Теорема 2.3.1, из которого следует, что
%
\begin{equation}
v(x) \le G^{-1} \left( G(\widetilde{C}) +  \dfrac{\widetilde{P}}{P_0} {\int \limits_{x_0}^{x} d\xi} \right) = G^{-1} \left( G(\widetilde{C}) + \dfrac{\widetilde{P}}{P_0}(x - x_0) \right) < \infty,
\label{eq:bihari}
\end{equation}
%
Неравенство (\ref{eq:bihari}) справедливо при всех $x \in [x_0, x_1)$.
Таким образом, из (\ref{eq:bihari}) следует ограниченность функции $v(x)$ на всем промежутке $[x_0; x_1)$, а именно
%
\begin{equation}
v(x) \le M = G^{-1} \left( G(\widetilde{C}) + \dfrac{\widetilde{P}}{P_0}(x_1 - x_0) \right).
\end{equation}
%
Заметим, что $\widetilde{C} \ge Q_0^2$, причем $\widetilde{C} = Q_0^2$ только при $u_0 = u_0' = 0$.
Это означает, что $G(s)$ определена в точке $\widetilde{C}$ для любого ненулевого решения $u(x)$.
Из ограниченности $v(x)$ следует, что решение $u(x)$ также является ограниченным на всем промежутке $[x_0; x_1)$:
%
\begin{equation}
|u(x)| \le \sqrt{\dfrac{\sqrt{M} - Q_0}{P_0}},~x \in [x_0, x_1).
\label{eq:estimstion}
\end{equation}
%
Подставляя оценку (\ref{eq:estimstion}) в равенство (\ref{eq:integrated}), получаем оценку сверху для производной $u_x(x)$, справедливую на полуинтервале $x \in [x_0; x_1)$.
Поскольку функции $u(x)$ и $u_x(x)$ непрерывны и ограничены на $[x_0; x_1)$, значения $u(x_1)$ и $u_x(x_1)$ конечны.
Следовательно, существует продолжение решения задачи Коши с начальными условиями $u(x_0)$, $u_x(x_0)$ на интервал, больший чем $[x_0; x_1)$, что противоречит исходному предположению.

Таким образом, доказана возможность продолжения решения на всю полупрямую $x > x_0$.
Для доказательства аналогичного факта для $x < x_0$ достаточно сделать замену $x \to -x$ и повторить вышеизложенные рассуждения.
\end{proof}

Из доказательства предложения \ref{prop:continuation}, в частности, следует, что если условия (а) и (б) выполняются не на всей числовой прямой, а только на некотором промежутке $[x_1;x_2]$, решение задачи Коши для уравнения (\ref{eq:stationary}) с произвольными начальными условиями не коллапсирует ни в какой точке промежутка $[x_1;x_2]$.
