\chapter{Регулярные и сингулярные решения}

В этой главе представлены некоторые результаты аналитического исследования уравнения (\ref{eq:stationary}) для стационарных состояний.
Для дальнейшего условимся о следующей терминологии.
Решение $u(x)$ уравнения (\ref{eq:stationary}) будем называть {\it сингулярным}, если для некоторой конечной точки $x_0 \in \mathbb{R}$ выполняется соотношение
%
$$\lim \limits_{x \to x_0} u(x) = \infty.$$
%
При этом мы будем говорить, что решение $u(x)$ {\it коллапсирует} в точке $x_0$.
Соответственно, решение $u(x)$ уравнения (\ref{eq:stationary}), не коллапсирующее ни в какой точке $\mathbb{R}$, будем называть {\it несингулярным} или {\it регулярным}.

\section{$P(x) > 0$: отсутствие сингулярных решений}

В случае, когда $P(x)$ --- положительная функция, при некоторых других ограничениях оказывается справедливым следующее утверждение.

\begin{proposition}
Пусть $\forall x \in \mathbb{R}$, функции $Q(x), P(x) \in C^1(\mathbb{R})$, причем
%
\begin{itemize}
\item[(a)] $P(x) \ge P_0 > 0$, $|P'(x)| \le \widetilde{P};$
\item[(б)] $Q(x) \ge Q_0$, $|Q'(x)| \le \widetilde{Q};$
\end{itemize}
%
тогда решение задачи Коши для уравнения (\ref{eq:stationary}) с произвольными начальными условиями $u(x_0) = u_0$, $u_x(x_0) = u_0'$ может быть продолжено на всю действительную ось $\mathbb{R}$.
\label{prop:continuation}
\end{proposition}
%
\begin{proof}
По теореме существования решений для ОДУ, существует такой интервал $\Delta = [x_0; x_1)$, что решение задачи Коши $u(x)$ уравнения (\ref{eq:stationary}) с начальными условиями $u(x_0) = u_0$, $u_x(x_0) = u_0'$ на нем существует, единственно и $u(x) \in C^2(\Delta)$.
Предположим, что $[x_0; x_1)$ --- это максимальный интервал существования решения, т.е. решение задачи Коши не может быть продолжено за точку $x = x_1$.
Умножив исходное уравнение на $4u_x$ и проинтегрировав в пределах $[x_0; x)]$, $x < x_1$, получим
%
\begin{eqnarray}
2u_{x}^2(x) + 2Q(x)u^2(x) - 2{\int \limits_{x_0}^x Q'(\xi)u^2(\xi)d\xi} + P(x)u^4(x) - \\
\nonumber - {\int \limits_{x_0}^x P'(\xi)u^4(\xi)d\xi} = 2(u_0')^2 + 2Q(x_0)u_0^2 + P(x_0)u_0^4 \equiv C,\label{eq:integrated}
\end{eqnarray}
%
Где $C$ --- некоторая известная постоянная.
Отбрасывая $u_x^2(x) \ge 0$ в левой части равенства, а также пользуясь ограниченностью снизу функций $Q(x)$ и $P(x)$ значениями $Q_0$ и $P_0$ соответственно, приходим к неравенству
%
\begin{equation}
2Q_0 u^2(x) + P_0 u^4(x) \le C + 2{\int \limits_{x_0}^x
Q'(\xi)u^2(\xi)d\xi} + {\int \limits_{x_0}^x P'(\xi)u^4(\xi)d\xi}.
\label{eq:bound}
\end{equation}
%
Производные $Q'(\xi)$ и $P'(\xi)$ заменим их оценками сверху: $Q'(\xi) \le \widetilde{Q}$, $P'(\xi) \le \widetilde{P}$, где $\widetilde{Q} \ge 0$, $\widetilde{P} \ge 0$.
Умножив обе части неравенства на $P_0 > 0$, получим, что
%
\begin{equation}
2Q_0 P_0 u^2(x) + P_0^2 u^4(x) \le P_0 C + 2P_0 \widetilde{Q}{\int\limits_{x_0}^x u^2(\xi)d\xi} + P_0 \widetilde{P}{\int \limits_{x_0}^x u^4(\xi)d\xi}.
\end{equation}
%
Обозначим $v(x) = (P_0 u^2(x) + Q_0)^2$, $v(x) \ge 0$.
Тогда
%
\begin{equation}
v(x) \le \widetilde{C} +  \dfrac{\widetilde{P}}{P_0}\int \limits_{x_0}^x w(v(\xi))~d\xi. \label{eq:tov}
\end{equation}
%
Здесь $\widetilde{C} = P_0 C + Q_0^2 \ge 0$, $\alpha = {2\widetilde{Q} P_0}/{\widetilde{P}} \ge 0$, а $w(v)$ определяется формулой
%
\begin{equation}
w(v)\equiv\alpha (\sqrt{v} - Q_0) + (\sqrt{v} - Q_0)^2.
\label{eq:defw}
\end{equation}
%
Введем в рассмотрение функцию
%
\begin{equation}
G(s) = \int \limits_{s_0}^s \dfrac{dv}{w(v)}.
\label{eq:G}
\end{equation}
%
Здесь $s_0 \ge Q_0^2$ --- произвольная постоянная, $s \ge s_0$.
Так как $w(v)$ положительна и монотонно возрастает, а интеграл
%
\begin{equation}
\int \limits_{s_0}^{+\infty} \dfrac{dv}{w(v)}
\end{equation}
%
расходится, функция $G(s)$ является положительной, монотонно возрастающей и неограниченной.
Это означает, что обратная функция $G^{-1}(r)$ определена при $r \ge 0$, монотонно возрастает и неограничена.
Сказанное позволяет применить к (\ref{eq:tov}) неравенство Бихари, \cite{Pachpatte}, Теорема 2.3.1, из которого следует, что
%
\begin{equation}
v(x) \le G^{-1} \left( G(\widetilde{C}) +  \dfrac{\widetilde{P}}{P_0} {\int \limits_{x_0}^{x} d\xi} \right) = G^{-1} \left( G(\widetilde{C}) + \dfrac{\widetilde{P}}{P_0}(x - x_0) \right) < \infty.
\label{eq:bihari}
\end{equation}
%
Неравенство (\ref{eq:bihari}) справедливо при всех $x \in [x_0, x_1)$.
Таким образом, из (\ref{eq:bihari}) следует ограниченность функции $v(x)$ на всем промежутке $[x_0; x_1)$, а именно
%
\begin{equation}
v(x) \le M = G^{-1} \left( G(\widetilde{C}) + \dfrac{\widetilde{P}}{P_0}(x_1 - x_0) \right).
\end{equation}
%
Заметим, что $\widetilde{C} \ge Q_0^2$, причем $\widetilde{C} = Q_0^2$ только при $u_0 = u_0' = 0$.
Это означает, что $G(s)$ определена в точке $\widetilde{C}$ для любого ненулевого решения $u(x)$.
Из ограниченности $v(x)$ следует, что решение $u(x)$ также является ограниченным на всем промежутке $[x_0; x_1)$:
%
\begin{equation}
|u(x)| \le \sqrt{\dfrac{\sqrt{M} - Q_0}{P_0}},~x \in [x_0, x_1).
\label{eq:estimstion}
\end{equation}
%
Подставляя оценку (\ref{eq:estimstion}) в равенство (\ref{eq:integrated}), получаем оценку сверху для производной $u_x(x)$, справедливую на полуинтервале $x \in [x_0; x_1)$.
Поскольку функции $u(x)$ и $u_x(x)$ непрерывны и ограничены на $[x_0; x_1)$, значения $u(x_1)$ и $u_x(x_1)$ конечны.
Следовательно, существует продолжение решения задачи Коши с начальными условиями $u(x_0)$, $u_x(x_0)$ на интервал, больший чем $[x_0; x_1)$, что противоречит исходному предположению.

Таким образом, доказана возможность продолжения решения на всю полупрямую $x > x_0$.
Для доказательства аналогичного факта для $x < x_0$ достаточно сделать замену $x \to -x$ и повторить вышеизложенные рассуждения.
\end{proof}

Таким образом, в случае $P(x) > 0$ показано отсутствие сингулярных решений уравнения (\ref{eq:stationary}).
Из доказательства предложения \ref{prop:continuation}, в частности, следует, что если условия (а) и (б) выполняются не на всей числовой прямой, а только на некотором промежутке $[x_1;x_2]$, решение задачи Коши для уравнения (\ref{eq:stationary}) с произвольными начальными условиями не коллапсирует ни в какой точке промежутка $[x_1;x_2]$.

\section{$P(x) < 0$, $Q(x) < 0$: все решения сингулярны}

Для дальнейших рассуждений нам понадобится несколько вспомогательных лемм.

\begin{lemma}
Все решения уравнения
%
\begin{equation}
v_{xx} - qv - pv^3 = 0,
\label{eq:singular}
\end{equation}
%
где $p,q > 0$ --- постоянные, сингулярны, за исключением нулевого решения.
\label{lemma:singular}
\end{lemma}
%
\begin{proof}
Решение задачи Коши для уравнения (\ref{eq:singular}) с начальными условиями $v(x_0) = v_0$, $v_x(x_0) = v_0'$ может быть записано в неявном виде
%
\begin{eqnarray}
&&\pm {\int \limits_{v_0}^{v} \dfrac{d\xi}{\sqrt{C + q{\xi}^2 + \dfrac{p}{2}{\xi}^4}}} = x - x_0; \\
&& \nonumber C = (v_0')^2 - qv_0^2 - \dfrac{p}{2}v_0^4,
\end{eqnarray}
%
(выбор знака в левой части зависит от начальных условий и значения $x$).
Интеграл в левой части равенства сходится при $v \to \infty$, а значит, существует такое значение $x$,
%
\begin{equation}
x_{collapse} = x_0 + {\int \limits_{v_0}^{\infty} \dfrac{d\xi}{\sqrt{C + q{\xi}^2 + \dfrac{p}{2}{\xi}^4}}},
\end{equation}
%
при приближении к которому $v(x)$ стремится к бесконечности.
\end{proof}

Вторая лемма, которая понадобится нам для доказательства основного утверждения этого раздела, это так называемая лемма о сравнении (Comparison Lemma) из работы \cite{AlfZez}.
Она формулируется следующим образом.

\begin{lemma}[О сравнении]
Пусть функции $u(x)$, $v(x)$, $x \in [a; b]$ являются решениями уравнений
%
\begin{eqnarray}
&& u_{xx} - g(x, u) = 0; \\
&& v_{xx} - f(x, v) = 0,
\end{eqnarray}
%
соответственно.
Пусть также выполняются следующие условия:
\begin{itemize}
\item[(а)] $f(x, \xi)$, $g(x, \xi)$ определены на $[a; b] \times [A; B]$ и для них выполняется локальное Липшица по переменной $\xi \in [A; B]$ ($A$, $B$, $b$ могут быть как конечными так и бесконечными);
\item[(б)] $g(x, \xi) \ge f(x, \xi)$ для всех $x \in [a; b]$, $\xi \in [A; B]$;
\item[(в)] $f(x, \xi)$ монотонная, неубывающая по $\xi \in [A; B]$ функция.
\end{itemize}
Далее, пусть $A < v(a) \le u(a) < B$ и $v_x(a) \le u_x(a)$, тогда верно, что $v(x) \le u(x)$ и $v_x(x) \le u_x(x)$, пока выполняется $A < v(x), u(x) < B$, или же на всем интервале $x \in [a; b]$ (если $\forall x \in [a; b]$ $A < v(x), u(x) < B$).
\label{lemma:comparison}
\end{lemma}

Теперь все готово для доказательства следующего утверждения.

\begin{proposition}
Пусть при любом $x \in \mathbb{R}$ выполняются условия $P(x) \le P_0 < 0$, $Q(x) \le Q_0 < 0$, тогда  все решения уравнения (\ref{eq:stationary}) сингулярны, за исключением нулевого решения.
\label{prop:singular}
\end{proposition}
%
\begin{proof}
Используем Лемму \ref{lemma:comparison} о сравнении.
Рассмотрим уравнение
%
\begin{equation}
v_{xx} + Q_0 v + P_0 v^3 = 0.
\end{equation}
%
Введем следующие обозначения:
%
\begin{eqnarray}
&& g(x, \xi) = -Q(x)\xi - P(x)\xi^3;\\
&& f(x, \xi) = f(\xi) = -Q_0 \xi - P_0 \xi^3.
\end{eqnarray}
%
Применим Лемму \ref{lemma:comparison} о сравнении к паре уравнений
%
\begin{eqnarray}
u_{xx} - g(x, u) = 0; \label{eq:less} \\
v_{xx} - f(x, v) = 0. \label{eq:greater}
\end{eqnarray}
%

В области $D_+ = \{ x \in \mathbb{R},~\xi \in (0; +\infty) \}$ имеем $f(x, \xi) \le g(x, \xi)$. Пусть $\tilde{u}(x)$ --- решение задачи Коши для уравнения (\ref{eq:less}) с начальными условиями $u(x_0) = u_0$, $u'(x_0) = u_0'$.
Выберем начальные условия задачи Коши для уравнения (\ref{eq:greater}): $v(x_0) = u(x_0) = u_0$, $v'(x_0) = u'(x_0) = u_0'$; $\tilde{v}(x)$ -- ее решение.
Пусть $u_0 > 0$, тогда возможны два случая:

(а) $u_0' \ge 0$.
Функция $\tilde{v}(x)$  монотонно возрастает (этот факт легко установить из фазового портрета уравнения (\ref{eq:greater})).
Исходя из леммы о сравнении, решение $\tilde{u}(x)$ ограничивает сверху решение $\tilde{v}(x)$, которое, в свою очередь, сингулярно.
Следовательно, решение $\tilde{u}(x)$ также является сингулярным.

(б) $u_0' < 0$.
Сделаем замену $\tilde{x} = -x$.
В этом случае решение $\tilde{v}(\tilde{x})$  также монотонно возрастает, а, исходя из леммы
о сравнении, $\tilde{u}(\tilde{x})$ ограничивает его сверху, следовательно, является сингулярным.

Аналогично, в области $D_- = \{x \in \mathbb{R}, \xi \in (-\infty; 0)\}$ справедливо неравенство $f(x,\xi) \ge g(x,\xi)$.
Путем аналогичных рассуждений устанавливается, что в области $D_-$ решение $u(x)$ также будет сингулярным.
\end{proof}

\section{$P(x)$ не знакоопределена}

Если $P(x)$ отрицательна хотя бы в одной точке $x_0 \in \mathbb{R}$, формальные асимптотические разложения предсказывают существование {\it двух однопараметрических семейств решений} уравнение (\ref{eq:stationary}), коллапсирующих в этой точке.

\subsection{Асимптотика сингулярных решений}

Построим эти асимптотические разложения.
Будем считать, что $P(x_0)=-1$ (выполнения этого условия можно добиться путем перенормировки независимой переменной), введем обозначение $\eta = x - x_0$ и предположим, что в окрестности точки $x = x_0$ справедливы разложения
%
\begin{eqnarray}
&& Q(x) = Q_0 + Q_1 \eta + Q_2 \eta^2 + \dots; \\
&& P(x) = -1 + P_1 \eta + P_2 \eta^2 + \dots.
\end{eqnarray}
%
Имеем
%
\begin{equation}
u_{\eta\eta} + (Q_0 + Q_1 \eta + Q_2 \eta^2 + \dots)u + (-1 + P_1 \eta + P_2 \eta^2 + \dots)u^3 = 0.
\end{equation}
%
Коллапсирующие в точке $x = x_0$ решения этого уравнения удовлетворяют условию $u(\eta) \to \pm \infty$, $\eta \to 0$.
Пусть $\eta$ стремиться к нулю {\it справа}, $\eta > 0$.
Сделаем замены $v(\eta) = \eta u(\eta)$, $\eta = e^{-t}$.
Получим
%
\begin{equation}
v_{tt} + 3v_t + 2v + e^{-2t}Q(t)v + P(t)v^3 = 0.
\label{eq:collapse_asympt}
\end{equation}
%
Главный член разложения определим, исходя из баланса членов при $2v$ и $-v^3$.
Соответственно имеем
%
\begin{equation}
V_0(t) = \pm \sqrt{2}.
\label{eq:V0}
\end{equation}
%
Найдем первую поправку к главному члену, $v(t) = \pm \sqrt{2} + V_1(t) + o(V_1(t))$.
Подставляя последнее выражение в уравнение (\ref{eq:collapse_asympt}), учитывая разложение для функций $Q(t)$, $P(t)$ и отбрасывая члены порядков, более высоких, чем $e^{-t}$, получаем
%
\begin{equation}
V_{1,tt} + 3V_{1,t} - 4V_1 = \mp 2 \sqrt{2} e^{-t},
\end{equation}
%
откуда $V_1(t) = \pm \frac{\sqrt{2}}{3} e^{-t}$.
Вторая, третья и четвертая поправки, $V_n$, $n = 2,3,4$, находятся аналогичным образом.
Для каждой из них соответствующее уравнение принимает вид
%
\begin{equation}
V_{n,tt} + 3V_{n,t} - 4V_n = C_n e^{-nt}.
\label{eq:Vn}
\end{equation}
%
Однако, если при $n = 2,3$ решения уравнения (\ref{eq:Vn}) имеют вид $V_n \sim e^{-nt}$, то в случае $n = 4$ показатель экспоненты в правой части совпадает с одним из корней характеристического многочлена для оператора, стоящего в правой части.
В этом случае решение уравнения (\ref{eq:Vn}) необходимо искать в виде $Ce^{-4t} - A_3 t e^{-4t}$.
Здесь постоянная $C$ может быть выбрана произвольным образом, а $A_3$ однозначно определяется коэффициентами разложений для $Q(t)$, $P(t)$.
Если константа $C$ фиксирована, на дальнейших шагах указанной процедуры соответствующие уравнения разрешаются однозначно.
Отметим, что замена <<$+$>> на <<$-$>> в соотношении (\ref{eq:V0}) приводит к замене знаков на противоположные у всех коэффициентов $A_n$, $n = 0, 1, \dots$, что естественно в силу инвариантности уравнения (\ref{eq:stationary}) относительно замены $u \to -u$.
Получаем
%
\begin{equation}
\pm v(t) = \sqrt{2} + A_0 e^{-t} + A_1 e^{-2t} + A_2 e^{-3t} + A_3 \cdot (-t) \cdot e^{-4t} + Ce^{-4t} + \dots
\end{equation}
%
Явные выражения для $A_0, \dots, A_3$ таковы:
%
\begin{eqnarray}
&& A_0 = \dfrac{\sqrt{2}}{3} P_1; \label{eq:A_0} \\
&& A_1 = \dfrac{\sqrt{2}}{3} P_2 + \dfrac{\sqrt{2}}{6} Q_0 + \dfrac{2\sqrt{2}}{9} P_1^2; \label{eq:A_1} \\
&& A_2 = \dfrac{2\sqrt{2}}{3} P_2 P_1 + \dfrac{7\sqrt{2}}{27} P_1^3 + \dfrac{\sqrt{2}}{6} Q_0 V_1 + \dfrac{\sqrt{2}}{4} Q_1 + \dfrac{\sqrt{2}}{2} P_3; \label{eq:A_2} \\
&& A_3 = -\dfrac{\sqrt{2}}{6} Q_1 P_1 - \dfrac{\sqrt{2}}{5} Q_2 - \dfrac{32 \sqrt{2}}{45} P_2 P_1^2 - \dfrac{3\sqrt{2}}{5} P_3 P_1 - \label{eq:A_3} \\ 
&& \nonumber - \dfrac{2\sqrt{2}}{15} P_2 Q_0 - \dfrac{2\sqrt{2}}{15} Q_0 P_1^2 - \dfrac{2\sqrt{2}}{5} P_4 - \dfrac{28\sqrt{2}}{135} P_1^4 - \dfrac{4\sqrt{2}}{15} P_2^2.
\end{eqnarray}
%

В случае, когда $\eta \to 0$ слева, $\eta < 0$, для построения подобных разложений необходимо сделать замены $v(\eta) = \eta u(\eta)$, $\eta = -e^{-t}$.
Формулы для коэффициентов $A_n$ при этом получаются такими же, что и в случае $\eta > 0$.
Окончательно для исходного решения $u(x)$ при $x \to x_0 \pm 0$ получаем
%
\begin{equation}
\pm u(x) = \dfrac{\sqrt{2}}{\eta} + A_0 + A_1 \eta + A_2 \eta^2 + A_3 \eta^3 \ln|\eta| + C\eta^3 + A_4\eta^4 \ln|\eta| + \dots,
\label{eq:final_asympt}
\end{equation}
%
где $A_0,\dots,A_1$ задаются формулами (\ref{eq:A_0})-(\ref{eq:A_3}), а дальнейшие коэффициенты $A_n$, $n > 3$ выражаются через $\{ Q_n \}$, $\{ P_n \}$ и произвольную константу $C$.

Суммируя сказанное, асимптотическое разложение (\ref{eq:final_asympt}) предсказывает существование двух однопараметрических семейств решений, коллапсирующих в точке $x_0$.
Эти семейства связаны между собой симметрией $u \to -u$.
При $x \to x_0 - 0$ решения одного из этих семейств стремятся к $+\infty$, а другого~---~к $-\infty$ соответственно.

\subsection{Существование однопараметрических семейств коллапсирующих решений}

Возможность построения асимптотического разложения (\ref{eq:final_asympt}) еще не является доказательством существования однопараметрических семейств коллапсирующих в точке $x_0$ решений.
Вместе с тем при некоторых дополнительных ограничениях на функции $Q(x)$, $P(x)$ этот факт удается доказать строго.
Для этого нам понадобится нижеследующая лемма.

\begin{lemma}[Об ограниченных решениях]
Пусть $f(t, z)$ --- функция, непрерывная по $t$ и непрерывно дифференцируемая по $z$, определенная для $t \ge t_0$ и $|z| < +\infty$ и обладающая следующими свойствами:
%
\begin{itemize}
\item[(а)] при $|z| < \rho$, $\rho > 0$, выполняются соотношения $|f(t,z)| < \eta_{\rho}(t)|z|$, причем $\eta_{\rho} \in L_1(t_0; +\infty)$;
\item[(б)] для любых $z_1$, $z_2$, таких, что $|z_{1,2}| < \rho$, $\rho > 0$, существует функция $\tilde{\eta}_{\rho}(t) \in L_1(t_0; +\infty)$, такая, что $|f(t, z_2) - f(t, z_1)| \le \tilde{\eta}_{\rho} |z_2 - z_1|$;
\item[(в)] при $|z| < \rho$, $\rho > 0$, выполняется соотношение $|f_z(t,z)| < \theta_{\rho}(t)|z|$, причем $\theta_{\rho} \in L_1(t_0; +\infty)$;
\item[(г)] для любых $z_1$ и $z_2$, таких, что $|z_{1,2}| < \rho$, $\rho > 0$, существует функция $\tilde{\eta}_{\rho}(t) \in L_1(t_0; +\infty)$, такая, что $|f_z(t, z_2) - f_z(t, z_1)| \le \tilde{\eta}_{\rho}(t) |z_2 - z_1|$.
\end{itemize}
%
Тогда для уравнения
%
\begin{equation}
z_{tt} - \alpha z_t + f(t,z) = 0, \quad \alpha > 0
\label{eq:bounded}
\end{equation}
%
справедливы следующие утверждения:
%
\begin{itemize}
\item[(A)] для любого ограниченного при $t \to +\infty$ решения $z(t)$ уравнения (\ref{eq:bounded}) существует $C \in \mathbb{R}$, такая, что $z(t) \to C$ при $t \to +\infty$;
\item[(Б)] для любого $C \in \mathbb{R}$ имеется единственное решение $Z(t, C)$ уравнения (\ref{eq:bounded}), определенное на промежутке $(t_C; +\infty)$, такое, что
%
\begin{equation}
Z(t;C)=C+o(1)\quad \mbox{при} \quad t \to +\infty;
\label{eq:bounded_asympt}
\end{equation}
%
\item[(В)] семейство решений $Z(t, C)$ является $C^1$-гладким по параметру $C$.
\end{itemize}
%
\label{lemma:bounded}
\end{lemma}
%
\begin{proof}
Докажем пункт (A).
Используя метод вариации постоянной, получаем, что решение уравнения (\ref{eq:bounded}) удовлетворяет равенству
%
\begin{equation}
z(t)=\varkappa_0 + \varkappa_1 e^{\alpha t}+\int \limits_{t_0}^t e^{\alpha \eta} \left( \int \limits_\eta^{+\infty} e^{-\alpha\xi}~f(\xi,z(\xi))~d\xi \right)d\eta.
\label{eq:bounded_solution}
\end{equation}
%
В силу условия (а), если $z(t)$ --- ограниченная при $t \to +\infty$, то интеграл
%
\begin{equation}
\int \limits_{t_0}^t e^{\alpha \eta} \left( \int \limits_\eta^{+\infty} e^{-\alpha\xi}~f(\xi,z(\xi))~d\xi \right)d\eta
\end{equation}
%
сходится.
Для ограниченных решений $\varkappa_1 = 0$, следовательно, $z(t)$ при $t \to +\infty$ стремиться к константе.
Пункт (А) доказан.

Докажем пункт (Б).
Сделаем замену $u(t) = z(t) - C$, где $C$ --- произвольное число.
Запишем уравнение (\ref{eq:bounded}) в виде системы
%
\begin{equation}
y_t = Ay + F(t,y),
\label{eq:hartman}
\end{equation}
%
где
%
\begin{equation*}
y =
\begin{pmatrix}
u \\
v
\end{pmatrix},~
A =
\begin{pmatrix}
0 & 1 \\
0 & \alpha
\end{pmatrix},~
F(t,y) =
\begin{pmatrix}
0 \\
f(t, u+C)
\end{pmatrix}.
\end{equation*}
%
Используем Теорему 9.1 из главы XII монографии \cite{Hartman}.
Она утверждает, что система (\ref{eq:hartman}) имеет решение, стремящееся к нулю на бесконечности, если выполнены следующие условия:
%
\begin{itemize}
\item[(1)] функция $F(t,y)$ непрерывна, причем $\|F(t,y)\| \le \lambda(t)$ при $t \in [t_0; +\infty)$, $\|y\| \le \rho$, где $\lambda(t) \in L_1(t_0, +\infty)$;
\item[(2)] при всех $g(t)={\rm col}(g_1(t),g_2(t))$, $g(t) \in L_1(t_0; +\infty)$ существует решение $y(t) \in L_0^{\infty}(t_0;+\infty)$ неоднородной системы
%
\begin{equation}
y_t = Ay + g(t);
\label{eq:hartman_n}
\end{equation}
%
\end{itemize}
%
(здесь и далее под нормой понимается евклидова норма в $\mathbb{R}^2$).

Во-первых, в силу (а) при $|u| \le \rho$ и $t > t_0$ имеет место соотношение $\|f(t, u, C)\| \le \rho \eta_{\rho}(t)$, причем $\eta_{\rho}(t) \in L_1(t_0; +\infty)$, следовательно, условие (1) теоремы выполняется.
Во-вторых, общее решение неоднородной системы уравнений (\ref{eq:hartman_n}) имеет вид:
%
\begin{equation}
u(t) = C_2 + \int  \limits_{t_0}^t \left( g_1(\eta) + e^{\alpha \eta} \left(C_1 - \int \limits_{\infty}^{\eta} e^{-\alpha \xi} g_2(\xi) d\xi \right) \right) d{\eta}; \\
\end{equation}
%
\begin{equation*}
\nonumber v(t) = u_t(t) - g_1(t).
\end{equation*}
%
Так как $g_{1,2}(t) \in L_1(t_0; +\infty)$, выбирая константы $C_1$, $C_2$ подходящим образом, можно получить решение, стремящееся к нулю при $t \to +\infty$, то есть выполняется условие (2).
Таким образом, условия цитируемой теоремы для системы (\ref{eq:hartman}) выполнены.
Это означает, что при любом $C$ уравнение (\ref{eq:bounded}) имеет решение $z(t)$, стремящееся в $C$ при $t \to +\infty$.

Докажем, что это решение единственно.
Пусть при одном и том же $C$ имеются два решения уравнения
%
\begin{equation}
u_{tt} - \alpha u_t + f(t, u + C) = 0.
\label{eq:bounded_aux}
\end{equation}
%
Тогда их разность $\Delta(t) = u_2(t) - u_1(t)$ удовлетворяет уравнению
%
\begin{equation}
\Delta_{tt} - \alpha \Delta_t + R(t) \Delta = 0
\label{eq:coppel}
\end{equation}
%
и граничному условию $\Delta \to 0$ при $t \to +\infty$.
Здесь
%
\begin{equation}
R(t) \equiv \frac{f(t,u_2(t)+C)-f(t,u_1(t)+C)}{u_2(t)-u_1(t)}.
\end{equation}
%
В силу условия (б) к уравнению (\ref{eq:coppel}) можно применит Теорему 11 из главы 3 монографии \cite{Coppel}.
Она утверждает, что существует гомеоморфизм между ограниченными решениями уравнения (\ref{eq:coppel}) и уравнения
%
\begin{equation}
\Delta_{tt} - \alpha \Delta_t = 0,
\label{eq:coppel_linear}
\end{equation}
%
причем (см. замечание после этой теоремы в \cite{Coppel}) в силу линейности возмущения, этот гомеоморфизм является линейными отображением.
Это означает, что только нулевое решение (\ref{eq:coppel}) удовлетворяет нулевому асимптотическому условию на бесконечности, т.е. $u_1(t) \equiv u_2(t)$.
Тем самым доказано существование семейства решений $Z(t, C)$, параметризуемого $C \in \mathbb{R}$, то есть пункт (Б) также доказан.

Для доказательства пункта (В) заметим, что производная
%
\begin{equation}
\frac{\partial Z}{\partial C}(t,C)\equiv\Theta(t,C)
\end{equation}
%
удовлетворяет уравнению (\ref{eq:coppel}), продифференцированному по $C$, причем \\ $\Theta(t, C)\to0$ при $t \to +\infty$.
Имеем
%
\begin{equation}
{\Theta}_{tt}-\alpha {\Theta}_t+f_z(t,u+C)\Theta+f_z(t,u+C) = 0.
\label{eq:coppel_diff}
\end{equation}
%
Вновь применяя Теорему 11 из главы 3 монографии \cite{Coppel} и, используя (в), (г), заключаем, что существует решение этого уравнения $\Theta(t,C)$, такое, что \\ $\Theta(t, C)\to0$ при $t \to +\infty$, причем $\Theta(t, C)$ непрерывна по параметру $C$.
Лемма доказана.
\end{proof}

Теперь перейдем к доказательству существования однопараметрических семейств коллапсирующих в точке $x_0$ решения уравнения (\ref{eq:stationary}).

\begin{proposition}
Пусть в уравнении (\ref{eq:stationary}) $P(x_0) = -1$; $\Omega$ --- некоторая окрестность точки $x_0$, а $Q(x) \in C^2(\Omega)$ и $P(x) \in C^4(\Omega)$.
Тогда существует два $C_1$-гладких однопараметрических семейства решений уравнения (\ref{eq:stationary}), соответствующих разложениям (\ref{eq:final_asympt}), коллапсирующих в точке $x = x_0$ (при подходе слева $x < x_0$) и связанных между собой симметрией $u \to -u$.
Каждое из этих семейств можно запараметризовать свободной переменной $C \in \mathbb{R}$ из разложения (\ref{eq:final_asympt}).
\label{prop:asymptotic}
\end{proposition}
%
\begin{proof}
В силу условий утверждения, справедливы равенства
%
\begin{eqnarray}
&& Q(x) = U_0 + U_1 \eta + U_2 \eta^2 + U_3 \eta^3 + Q_1(\eta)\eta^4; \\
&& P(x) = -1 + V_1 \eta+ V_2 \eta^2 + V_3 \eta^3 + V_4 \eta^4 + P_1(\eta)\eta^5,
\end{eqnarray}
%
где $\eta = x - x_0$ и $\widetilde{Q}(\eta), \widetilde{P}(\eta) \in C(\Omega)$.
Для доказательства существования семейства, соответствующего знаку <<$+$>> в (\ref{eq:final_asympt}), сделаем замену
%
\begin{equation}
u(x) = \frac{\sqrt{2}}\eta + A_0 + A_1 \eta + A_2 \eta^2 + A_3 \eta^3 \ln|\eta| + z(\eta)\eta^3,
\end{equation}
%
где $z(\eta)$ --- новая неизвестная функция.
Коэффициенты $A_0, A_1, A_2$ и $A_3$ выбираются в соответствии с формулами (\ref{eq:A_0})-(\ref{eq:A_3}), то есть, таким образом, чтобы обратить в ноль множители при степенях $\eta^{-2}$, $\eta^{-1}$, $\eta$ и $\eta^0$.
Непосредственная проверка показывает, что при таком выборе $A_k$, $k = 0,1,2,3$ из уравнения (\ref{eq:stationary}) вытекает следующее уравнения для $z(\eta)$:
%
\begin{equation}
z_{\eta\eta} + \frac{6}{\eta} z_\eta + g(\eta, z) = 0,
\label{eq:collapse_eq}
\end{equation}
%
где $g(\eta, z)$ --- полином третьей степени относительно $z$ и $g(\eta, z) \sim \frac{\ln(-\eta)}{\eta}$ при $\eta \to -0$ и фиксированном $z$.
Замена $\eta = -e^{-t}$ переводит точку $\eta = 0$ в $t = +\infty$, а уравнение (\ref{eq:collapse_eq}) в уравнение
%
\begin{equation}
z_{tt} - 5z_t - f(t,z) = 0,
\label{eq:collapse_eq_subs}
\end{equation}
%
причем $f(t,z) \sim te^{-t}$ при $t \to +\infty$.
Условия на $f(t,z)$ позволяют применить к уравнению (\ref{eq:collapse_eq_subs}) Лемму \ref{lemma:bounded} об ограниченных решениях.
Исходя из нее заключаем, что все ограниченные при $t \to +\infty$ решения уравнения (\ref{eq:collapse_eq_subs}) стремятся к некоторой константе $C$ при $t \to +\infty$.
Более того, эти решения образуют $C^1$-гладкое семейство.
Возвращаясь к уравнению (\ref{eq:collapse_eq}) и, далее, к (\ref{eq:stationary}), получаем искомое утверждение.
Существование второго семейства решений, соответствующего знаку <<$-$>> в (\ref{eq:final_asympt}), следует из инвариантности уравнения (\ref{eq:stationary}) относительно замены $u \to -u$.
\end{proof}

{\it Комментарий:} аналогичные однопараметрические семейства коллапсирующих решений существуют также и {\it справа} от точки $x = x_0$.

Множество несингулярных решений в случае, когда $P(x)$ отрицательна на некоторых промежутках числовой прямой, может быть устроено достаточно сложным образом.
В работе \cite{AlfAvr} было установлено, что если $P(x) \equiv -1$ и $Q(x)$ удовлетворяет некоторым дополнительным условиям, каждому несингулярному решению уравнения (\ref{eq:stationary}) можно поставить в соответствие бесконечную в обе стороны последовательность символов конечного алфавита, причем соответствие между множеством несингулярных решений и множеством символов является гомеоморфизмом.
В следующей главе мы изложим основные идеи подхода, использованного в работе \cite{AlfAvr}, и, затем, применим его для случая $Q(x) \equiv \omega = const$, $P(x) = \alpha + \cos 2x$.