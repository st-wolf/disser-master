\intro

% TODO: разбавить научпопом
% TODO: совершенно не понятно, что такое FS и FGS

Начиная с 90-х годов прошлого столетия, нелинейное уравнение Шредингера (НУШ) с дополнительной пространственной неавтономностью продолжает оставаться объектом пристального изучения.
Интерес к этому классу уравнений во многом обусловлен успехами в нелинейной оптики и исследованиям в области конденсата Бозе-Эйнштейна (БЭК).
БЭК --- это состояние вещества, возникающее при сверхнизких температурах, существование которого было предсказано в 20-е года XX века.
В 1995 г. БЭК был получен экспериментально \cite{Anderson}.
Оказалось, что динамика БЭК хорошо описывается уравнением шредингеровского типа с неавтономностью в виде дополнительного внешнего потенциала.
В контексте теории БЭК это уравнение носит название {\it уравнение Гросса-Питаевского}.
В дипломной работе рассматривается одномерный случай уравнения Гросса-Питаевского, которое имеет вид
%
\begin{equation}
i \Psi_t + \Psi_{xx} - U(x)\Psi + P(x)|\Psi|^2 \Psi = 0.
\label{eq:GPE}
\end{equation}
%
Здесь $\Psi(x, t)$ --- макроскопическая волновая функция конденсата, $U(x)$ соответствует потенциалу ловушки, удерживающей конденсат, а $P(x)$ описывает характерную длину межатомных взаимодействий частиц конденсата.
Используя явление так называемого резонанса Фешбаха, эта длина в эксперименте может быть сделана переменной, в частности изменяющейся периодически в пространстве \cite{Theis}.
Альтернативный подход основывается на использовании различных магнитных решеток, в которые погружается конденсат \cite{Jose}.
В этом случае говорят о взаимодействии конденсата с {\it нелинейной решеткой} (более подробно см. в обзоре \cite{Kartashov}).
Стоит отметить, что $P(x)$ может быть как знакопостоянной, так и знакопеременной функцией.
Особое внимание в литературе уделяется двум модельным случаям: $P(x) \equiv 1$ (случай межатомного притяжения) и $P(x) \equiv -1$ (случай межатомного отталкивания).

Потенциал $U(x)$ в случае оптической ловушки также моделируется периодической функцией.
В этом случае говорят о {\it линейной решетке}, удерживающей конденсат.
Обсуждение соответствующих физических принципов можно найти в работе \cite{Pitaevskii}, а обзор результатов исследования уравнения Гросса-Питаевского с периодическим (линейным) потенциалом --- в работе \citep{Brazhnyi}.
В дальнейшем мы будем называть функцию $U(x)$ {\it линейным потенциалом}, а функцию $P(x)$ --- {\it нелинейным потенциалом}.

Стоит отметить, что уравнение (\ref{eq:GPE}) является широко востребованным в самых разных областях физики.
В оптике уравнение (\ref{eq:GPE}) описывает распространение света в планарных волноводах.
Пространственная модуляция в этом случае достигается за счет присадок неоднородной плотности, встроенных в волновод и усиливающих нелинейные свойства.
Таким образом, описание и исследование решений уравнения (\ref{eq:GPE}) для различных $U(x)$, $P(x)$ является актуальной математической задачей.

При исследовании задач, приводящих к уравнению (\ref{eq:GPE}), с физической точки зрения важную роль играют так называемые {\it стационарные моды}.
Им соответствуют решения вида $\Psi(x, t) = u(x) e^{-i \omega t}$.
Если $u(x)$ --- действительная функция, она удовлетворяет обыкновенному дифференциальному уравнению
%
\begin{equation}
u_{xx} + Q(x)u + P(x)u^3 = 0,
\label{eq:stationary}
\end{equation}
%
где $Q(x) = \omega - U(x)$.
Естественным (опять же с физической точки зрения) является {\it условие локализации} стационарной моды
%
\begin{equation}
\lim \limits_{x \to \pm \infty} u(x) = 0.
\label{eq:localization}
\end{equation}
%
Вместе с тем в литературе рассматриваются и другие типы стационарных мод, в частности, пространственно периодические и квазипериодические структуры.
Однако большая часть литературы посвящена именно стационарным решениям.
При этом внимание зачастую уделяется лишь простейшим видам локализованных стационарных мод, в частности так называемым {\it fundamental solitons} (FS) или {\it fundamental gap solitons} (FGS), которых отличает их характерная колоколообразная форма \cite{Kartashov}.

Второе важное условие, которое естественно потребовать от стационарные моды, это ее устойчивость.
Это означает, что достаточное малые возмущения стационарной моды не возрастают со временем.
Удовлетворение этому требованию важно для физических приложений, поскольку зачастую только устойчивые решения могут быть обнаружены в эксперименте и в дальнейшем использованы на практике.
Одной из глобальных задач, связанных с уравнением Гросса-Питаевского, является поиск и классификация новых локализованных стационарных мод и исследование их устойчивости.

Для описания множества стационарных мод необходимо детальное исследование множества решений уравнения (\ref{eq:stationary}).
Попытка такого исследования в случае, когда $P(x) \equiv -1$ (отталкивающие межатомные взаимодействия) и $Q(x)$ --- ограниченная периодическая функция, была сделана в работе \cite{AlfAvr}.
Основная идея работы \cite{AlfAvr} заключалась в том, что ``большая часть'' решений уравнения (\ref{eq:stationary}) уходит на бесконечность в некоторой конечной точке числовой прямой, т.е. $\exists x_0 \in \mathbb{R}$, что
$$\lim \limits_{x \to x_0} u(x) = \infty.$$
В дальнейшем мы будем называть такие решения {\it сингулярными}.
Далее, оказалось, что множество ``оставшихся'' решений, определенных на всей числовой прямой, при некоторых условиях можно полностью описать, используя методы символической динамики.
Более точно, эти решения могут быть поставлены во взаимно-однозначное соответствие с бесконечными в обе стороны последовательностями из символов некоторого конечного алфавита ({\it кодами} решений).
Из этого множества кодов можно выделить коды, соответствующие локализованным или периодическим решениям.
В работе \cite{AlfAvr} был проведен детальный анализ случая
$$Q(x) = \omega - A \cos 2x,$$
и была указана область на плоскости параметров $(\omega, A)$, где все определенные на всей числовой прямой решения можно закодировать бесконечными последовательностями из символов ``$0$'',  ``$+$'', ``$-$''.
Локализованным модам при этом соответствуют коды, которые содержат лишь конечное число ненулевых символов.

Задачей, которая решается в ходе дипломной работы, является применение аналогичного подхода для случая, когда функция $P(x)$ является знакопеременной.
В частности, интересно посмотреть простейший случай периодической модуляции, когда $P(x) = \alpha + \cos 2x$, а $U(x) \equiv 0$.
Этот случай уже обсуждался в литературе (см. работу \cite{Malomed}) при рассмотрении солитонов в оптических решетках.
В ней подробно исследовалась устойчивость простейшего класса стационарных мод --- FS решений, и была показана устойчивость FS для большинства значений параметров.
Будет полезным подтвердить основные результаты этой работы и несколько расширить их.
В заключении вводной части резюмируем основные направления исследований.

\paragraph{Цели дипломной работы.}
\begin{itemize}
\item[1.] Обобщение результатов работы \cite{AlfAvr} на более широкий класс функций $Q(x)$, $P(x)$.
\item[2.] Поиск и разработка универсальных методов определения устойчивости стационарных мод.
\item[3.] Применение этих техник при рассмотрении модельного случая уравнения (\ref{eq:GPE}), когда $U(x) \equiv 0$, $P(x) = \alpha + \cos 2x$; поиск новых локализованных решений и исследование их устойчивости.
\end{itemize}