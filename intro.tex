\intro

% TODO: разбавить научпопом
% TODO: разбавить маломедом

Начиная с 90-х годов прошлого столетия, нелинейное уравнение Шредингера (НУШ) с дополнительной пространственной неавтономностью продолжает оставаться объектом пристального изучения.
Интерес к этому классу уравнений во многом обусловлен успехами в нелинейной оптики и исследовании конденсата Бозе-Эйнштейна (БЭК).
БЭК --- это состояние вещества, возникающее при сверхнизких температурах, существование которого было предсказано в 20-е года XX века.
В 1995 г. БЭК был получен экспериментально \cite{Anderson}.
Оказалось, что динамика БЭК хорошо описывается уравнением шредингеровского типа с неавтономностью в виде дополнительного внешнего потенциала.
В контексте теории БЭК это уравнение носит название {\it уравнение Гросса-Питаевского}.
В дипломной работе рассматривается одномерный случай уравнения Гросса-Питаевского, которое имеет вид
%
\begin{equation}
i \Psi_t + \Psi_{xx} - U(x)\Psi + P(x)|\Psi|^2 \Psi = 0.
\label{eq:GPE}
\end{equation}
%
Здесь $\Psi(x, t)$ --- макроскопическая волновая функция конденсата, $U(x)$ соответствует потенциалу ловушки, удерживающей конденсат, а $P(x)$ описывает характерную длину межатомных взаимодействий частиц конденсата.
Используя явление так называемого резонанса Фешбаха, эта длина в эксперименте может быть сделана переменной, в частности изменяющейся периодически в пространстве \cite{Theis}.
В этом случае говорят о взаимодействии конденсата с {\it нелинейной решеткой} (более подробно см. в обзоре \cite{Kartashov}).
Альтернативный подход основывается на использовании различных магнитных решеток, в которые погружается конденсат \cite{Jose}.
Стоит отметить, что функция $P(x)$ может быть как знакопостоянной, так и знакопеременной функцией.
Особое внимание в литературе уделяется двум модельным случаям: $P(x) \equiv 1$ (случай межатомного притяжения) и $P(x) \equiv -1$ (случай межатомного отталкивания).

Потенциал $U(x)$ в случае оптической ловушки также моделируется периодической функцией.
В этом случае говорят о {\it линейной решетке}, удерживающей конденсат.
Обсуждение соответствующих физических принципов можно найти в работе \cite{Pitaevskii}, а обзор результатов исследования уравнения Гросса-Питаевского с периодическим (линейным) потенциалом --- в работе \citep{Brazhnyi}.
В дальнейшем мы будем называть функцию $U(x)$ {\it линейным потенциалом}, а функцию $P(x)$ --- {\it нелинейным потенциалом}.

Стоит отметить, что уравнение (\ref{eq:GPE}) является широко востребованным в самых разных областях физики.
В оптике уравнение (\ref{eq:GPE}) описывает распространение света в планарных волноводах (где $x$ --- поперечная пространственная координата, а $t$ заменятся на $z$ --- расстояние распространения).
Пространственная модуляция в этом случае достигается за счет присадок неоднородной плотности, встроенных в волновод и усиливающих нелинейные свойства.
Таким образом описание и исследование решений уравнения (\ref{eq:GPE}) для различных $Q(x)$, $P(x)$ является актуальной математической задачей.

При исследовании задач, приводящих к уравнению (\ref{eq:GPE}), с физической точки зрения важную роль играют так называемые {\it стационарные моды}.
Им соответствуют решения вида $\Psi(x, t) = e^{-i \omega t}$.
Если $u(x)$ --- действительная функция, она удовлетворяет обыкновенному дифференциальному уравнению
%
\begin{equation}
u_{xx} + Q(x)u + P(x)u^3 = 0,
\label{eq:stationary}
\end{equation}
%
где $Q(x) = \omega - U(x)$.
В этом случае (опять же с физической точки зрения) естественным является {\it условие локализации} стационарной моды
%
\begin{equation}
\lim \limits_{x \to \pm \infty} u(x) = 0.
\label{eq:localization}
\end{equation}
%
Вместе с тем в литературе рассматриваются и другие типы стационарных мод, в частности, пространственно периодические и квазипериодические структуры.

Второе важное условие, которое естественно потребовать от стационарные моды, это ее устойчивость.
Это означает, что достаточное малые возмущения стационарной моды не возрастают со временем.
Это важно для физических приложений, поскольку зачастую только устойчивые решения могут быть обнаружены в эксперименте и, в дальнейшем, использованы на практике.